\documentclass{article}
\usepackage[utf8]{inputenc}
\usepackage{amsmath}
\usepackage{listings}

\usepackage{amsthm}
\usepackage{graphicx}
\usepackage{upgreek}
\usepackage{algorithm}
\usepackage{algpseudocode}


\date{20 February 2019}
\marginparwidth 0.5in 
\oddsidemargin 0.25in 
\evensidemargin 0.25in 
\marginparsep 0.25in
\topmargin 0.25in 
\textwidth 6in \textheight 8 in
\newtheorem{question}{Question: }
\theoremstyle{case}
\newtheorem{case}{Case}
\graphicspath{ {images/} }
\begin{document}
\author{Sai Vineet Reddy Thatiparthi}

 \title{%
  Advanced Machine Learning \\
  \large Homework 3}
\maketitle
\begin{enumerate}
    \item [1.] \textbf{For the following figure, (a) Is A independent of C given B and D in each
of the 3 cases? (b) Is B conditionally independent of D given A and C}
\end{enumerate} 
\begin{proof} 
I will be using d-separation to figure this out.\\ \\
\textbf{(a) (i)} From the figure, we are told that $B$ and $D$ are observed. There are two paths from $A$ to $C$, which are $ABC$ and $ADC$. Let's look at the path $ABC$ - here, the figure is in a head-tail configuration, with $B$ being observed. This means that, $B$ will block the path from $A$ to $C$. The same thing happens in the other path too. Since both paths are blocked, we can conclude that $A$ is conditionally independent of $C$ given $B, D$.\\ \\
\textbf{(ii)} From the figure, $B$ and $C$ are observed. There are two paths from $A$ to $C$, which are $ABC$ and $ADC$. Let's look at the path $ABC$ - here, the figure is in a head-head configuration, where $B$ is being observed. $B$ will block the path from $A$ to $C$. In an H-H configuration, a path becomes unblocked if the node is observed. This means that $B$ will not block the path. Hence, A is dependent on C given B and D.\\ \\
\textbf{(b)(i)} From the figure, we are told that $A$ and $C$ are observed. There are two paths from $B$ to $D$, which are $DCB$ and $DAB$. Let's look at the path $DCB$ - here, the figure is in a head-head configuration. Since C is being observed, the path becomes unblocked. Hence, we can safely say that B is not independent of D given A and C. \\ \\
\textbf{(b)(ii)} From the figure, we are told that $A$ and $C$ are observed. There are two paths from $B$ to $D$, which are $DCB$ and $DAB$. Let's look at the path $DAB$ - it is in a tail-tail configuration with A being observed. This means that A will block the path from D to B. Now coming to path DCB - it it is also in a tail-tail configuration with C being observed. This means that C will block the path from D to B. Hence, since both paths are blocked, we get that D is independent of B given A and C.
\end{proof}

\begin{enumerate}
    \item [2.] \textbf{Consider the problem discussed in class (Notes set 5-bn.pdf; the example
with heat disease etc.). Infer the probability that an individual has heart
disease given that the indivual has high blood pressure, has ahealth dietg
and exercise regularly}
\end{enumerate}
\begin{proof}
The MSE of the different polynomials will be displayed at the bottom of the plotted functions on the graph itself. These graphs are also the same graphs as sub-question D.

\end{proof}
\begin{enumerate}
    \item [3.] \textbf{Induce the Bayesian network topology for the following problem: Two
algorithms, A1 and A2 continually observe intrusions into a computer
network. Suppose, A1 indicates an intrusion while A2 does not. Sometimes
an intrusion is detected because the system administrator upgraded a
previous version of an application. Is there an intrusion? Make as many
reasonable assumptions as required.}
\end{enumerate} 
\begin{proof} 
The features that we will assume are present and affecting our computer network are - 
\begin{itemize}
  \item OS Version
  \item Software Version
  \item Internet Access
  \item USB Access
  \item Intrusion
\end{itemize}
The probability joint distribution will be as follows -
\begin{table}[]
\begin{tabular}{ll}
\textbf{IA} & \textbf{USB} \\
0.8 & 0.9
\end{tabular}
\end{table}
\end{proof}

\end{document}